
% uso de cores
%
%\textcolor{red}{This is in red.}
%\color{blue} This is switched to blue.  \color{black}  This is switched back to black.
%\colorbox{yellow}{This is in a yellow box.}

% new commands

%ignorar bloco de linhas
\newcommand{\ig}[1]{}

% Escrito a mao
\newcommand{\hw}[1]
{{\ensuremath{\mathcal  #1}}}

% I - Indutor
\newcommand{\ind}
{\ensuremath{\mathcal{I\hspace{-.05em}}}\xspace}

% C - Classificador
\newcommand{\class}
{\ensuremath{\mathcal{C\hspace{-.05em}}}\xspace}

% D - Distribui\c{c}\~ao
\newcommand{\dist}
{\ensuremath{\mathcal{D\hspace{-.05em}}}\xspace}

% MLC++
\newcommand{\mlc}
  {\ensuremath{\mathcal{MLC\hspace{-.05em}\raisebox{.4ex}{\tiny\bf ++}}}\xspace}

% C++
\newcommand{\cplusplus}
  {\ensuremath{C\hspace{-.05em}\raisebox{.4ex}{\tiny\bf ++}}\xspace}

% C++
\newcommand{\cpp}
  {\cplusplus}

% BibTeX
\def\BibTeX{{\rm B\kern-.05em{\sc i\kern-.025em b}\kern-.08em
    T\kern-.1667em\lower.7ex\hbox{E}\kern-.125emX}\xspace}

%% LaTeX
%\def\LaTeX{{\rm L\kern-.05em{\sc a\kern-.025em}\kern-.08em
%    T\kern-.1667em\lower.7ex\hbox{E}\kern-.125emX}\xspace}

% Nro
\newcommand{\nro}
 {\scriptsize $^{o}\,$\normalsize}

% IA
\newcommand{\IA}
 {Intelig�ncia Artificial \xspace}

% Nra
\newcommand{\nra}
  {\scriptsize $^{\b{a}}\,$\normalsize}

% e.g.
\newcommand{\eg}
  {\emph{ e.g.}\/\xspace}

% trademark
\newcommand{\TM}
  {\footnotesize\ensuremath{^{\rm TM}}\normalsize\xspace}
\newcommand{\tm}
  {\TM}

% Palavras em ingles
\newcommand{\engl}[1]
  {\emph{#1}}
%  {\selectlanguage{english}\emph{#1}\selectlanguage{brazil}}

% URL
%\newcommand{\url}[1]
%  {{\tt #1}\xspace}

\newcommand{\ip}[2]
  {(#1, #2)}

\newcommand{\seq}[3][X,1,n]
  {\lbrace #1_{#2},\ldots,\,#1_{#3} \rbrace}

\newcommand{\cart}[3][X,1,n]
  {#1_{#2} \times \ldots \times #1_{#3}}

% entrada de indice simples
\newcommand{\idxa}[1]
  {\emph{ #1}\index{#1}}

% entrada de indice dupla
\newcommand{\idxb}[2]
  {\emph{ #1 #2}\index{#1!#2}}

\newcommand{\HRule}{\rule{\linewidth}{1mm}}

% entrada de indice tripla
\newcommand{\idxc}[3]
  {\emph{ #1 #2 #3}\index{#1!#2!#3}}

% LaTeX para DOS
%\newcommand{\figura}[6]
%{\begin{figure}[htb]
%  \setlength{\unitlength}{1.0cm}
%  \centering
%  \begin{picture}(#1, #2)(0, 0)
%    \special{isoscale ./#3.#4, #1cm #2cm}
%  \end{picture}
%  \caption{#6}
%  \label{#5}
% \end{figure}
%}

% PCTeX
% figura x-size y-size filename extension label caption
%          1      2       3        4         5      6

%\newcommand{\figura}[6] % WMF
%{\begin{figure}[htb]
%   \vspace*{1cm}
%   \setlength{\unitlength}{1.0cm}
%   \centering
%   \begin{picture}(#1, #2)(0, 0)
%     \special{#4:./#3.#4 x=#1cm y=#2cm}
%   \end{picture}
%   \caption{#6}
%   \label{#5}
% \end{figure}
%}


% figura scale filename extension label caption
%          1      2       3        4         5

\newcommand{\figura}[6]
{\begin{figure}[htb]
   \vspace*{1cm}
   \setlength{\unitlength}{1.0cm}
   \centering
     \includegraphics[scale=#1]{./#2.#3}
\begin{center}
\parbox{.5\linewidth}{\footnotesize \sf \caption #6 {#5}  \label{#4}}
\end{center}

 \end{figure}
}

%-------------------------------------------------------------------------------------------
%\newtheorem{teorema}{Teorema}[section]
%\newtheorem{corolario}{Corolario}[section]
%\newtheorem{lema}{Lema}[section]
%\newtheorem{algo}{Algoritmo}[section]
%\newtheorem{defi}{Defini\c c\~ao}[section]
%\newtheorem{exem}{Exemplo}[section]
%-------------------------------------------------------------------------------------------




% bullet
\newcommand{\bb}
  {\ensuremath{\bullet}}

% ECLE
\newcommand{\ecle}
  {\texttt{ECLE}\xspace}

\newcommand{\snifferecle}
  {\texttt{SnifferECLE}\xspace}

% APRIORI
\newcommand{\apriori}
  {$APRIORI$\xspace}

% C4.5
\newcommand{\cfourfive}
  {$C4.5$\xspace}

% C4.5-rules
\newcommand{\cfourfiver}
  {\ensuremath{\mathcal{C}4.5-rules}\xspace}

% CN2
\newcommand{\cntwo}
  {$CN2$\xspace}

% RS
\newcommand{\rs}
  {\ensuremath{\mathcal{RS}}\xspace}


%PBM
\newcommand{\pbm}
   {$\mathcal{PBM}$\xspace}

% Marca1
\newcommand{\mone}
  {$^{\bf{\dag}}$}

% Marca2
\newcommand{\mtwo}
  {$^{\bf{\star}}$}

% i.e.
\newcommand{\ie}
  {\emph{ i.e.}\/\xspace}

% Espaco
\newcommand{\esp}
  {\hspace{0.3em}}

%% Definition for Big letter at the beginning of a paragraph
%%
\def\PARstart#1#2{\begingroup\def\par{\endgraf\endgroup\lineskiplimit=0pt}
    \setbox2=\hbox{\uppercase{#2} }\newdimen\tmpht \tmpht \ht2
    \advance\tmpht by \baselineskip\font\hhuge=cmr10 at \tmpht
    \setbox1=\hbox{{\hhuge #1}}
    \count7=\tmpht \count8=\ht1\divide\count8 by 1000 \divide\count7 by\count8
    \tmpht=.001\tmpht\multiply\tmpht by \count7\font\hhuge=cmr10 at \tmpht
    \setbox1=\hbox{{\hhuge #1}} \noindent \hangindent1.05\wd1
    \hangafter=-2 {\hskip-\hangindent \lower1\ht1\hbox{\raise1.0\ht2\copy1}%
    \kern-0\wd1}\copy2\lineskiplimit=-1000pt}

\if@technote\def\PARstart#1#2{#1#2}\fi     % if technical note, disable it
\if@draftversion\def\PARstart#1#2{#1#2}\fi % if draft, disable it

\newcommand{\fivextwo}
  {$5\times2$ \emph{Cross-Validation}\xspace}

\newcommand{\tencross}
	{$10$\emph{-Fold Cross-Validation}\xspace}

% Discover
\newcommand{\discover}
  {{\sc discover}\xspace}

\newcommand{\sniffer}
  {{\sc sniffer}\xspace}

\newcommand{\basename}[1]
  {{\small \textsf{#1}}\xspace}

\newcommand{\breast}  {\basename{breast}}
\newcommand{\german}  {\basename{german}}
\newcommand{\glass}   {\basename{glass}}
\newcommand{\heart}   {\basename{heart}}
\newcommand{\nursery} {\basename{nursery}}
\newcommand{\pima}    {\basename{pima}}

% \newcommand{\crossover} {\emph{crossover}\xspace}

% perl
\newcommand{\perl}
  {{\sc perl}\xspace}

\newcommand{\mx}{\textbf{\textsf{\textbf{x}}}\xspace}

% Backref com coloca�o de "Citado na p�ina..."
\renewcommand*{\backref}[1]{}
\renewcommand*{\backrefalt}[4]{
    \ifcase #1
        N�o citado no texto.
    \or
        Citado na p�gina~#2.
    \else
        Citado nas p�ginas #2.
    \fi
}



